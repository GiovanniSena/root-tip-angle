% Appendix A


\chapter{On the data set}

The data set used can be found on [INSERT REFERENCE GITHUB HERE].

%----------------------------------------------------------------------------------------
%----------------------------------------------------------------------------------------
\section{Suggestions for future data acquisition}

We will give some suggestions on how to improve the quality of the images in the future.


%----------------------------------------------------------------------------------------
\subsection{Stable conditions in the experiment}

In general, one should ensure to keep the conditions over one experiment stable as far as possible. It should be noted that this is far harder to achieve in a dynamic system as the one to capture electrotropism compared to a gravitropism setup usually done in an agar gel. 
This includes a constant number of tubes and roots, constant water level or surface and no external movement to the roots or the tubes to keep the noise distribution as constant as possible. Roots that are affected by change in reflections or illumination (eg by people walking into the room) have been proven to be very hard to handle and are often lost in the preprocessing step. 
Keeping the medium clean of dirt and bubbles as far as possible also will improve the preprocessing step. 


\subsection{No objects interfering with the object of interest}

The experimentalist should make sure that there are no objects interfering with the object of interest, be it other roots or any other similar looking objects that are hard to discern even by eye, throughout the experiment.

%llumination different — challenge to get background right
%local thresholds maybe?


%----------------------------------------------------------------------------------------
\subsection{Increase of the contrast}

It is advisable to work towards achieving a higher contrast in the images, so roots can be clearly distinguished from the background, also by eye.


%%----------------------------------------------------------------------------------------
%\subsection{More focus on images}


%----------------------------------------------------------------------------------------
\subsection{Higher resolution}


As the highly pixeled nature of the images did cause problems also in the angle computation, it is recommendable to try to increase the resolution of the images. We could also try to zoom into the roots or even one single root so our actual object(s) of interest take a bigger fraction in the images instead of wasting resolution on things that are of no interest. 

It could also be the compression step after taking the images that might cause or contribute to the low resolution of the images. One could attempt to user other formats to save the images. 

As a last suggestion, different cameras could be compared to see if it does have an effect on the quality of the images. 

%Better camera, less waste of resolution
%improve spatial resolution, other format? no compression?



%----------------------------------------------------------------------------------------
%----------------------------------------------------------------------------------------
%----------------------------------------------------------------------------------------
\chapter{Manual of \textit{RootSkel}}


%%%%%%%%%%%%%%%%%%%%%%%%%%%%%%%%%%%%%%%%%%%%%
%\chapter{Data}
%
%The data was taken by a ... camera using a raspberry Pi.
%Insert details regarding data collection.
%
%\chapter{Frequently Asked Questions} % Main appendix title
%
%\label{AppendixA} % For referencing this appendix elsewhere, use \ref{AppendixA}
%
%\section{How do I change the colors of links?}
%
%The color of links can be changed to your liking using:
%
%{\small\verb!\hypersetup{urlcolor=red}!}, or
%
%{\small\verb!\hypersetup{citecolor=green}!}, or
%
%{\small\verb!\hypersetup{allcolor=blue}!}.
%
%\noindent If you want to completely hide the links, you can use:
%
%{\small\verb!\hypersetup{allcolors=.}!}, or even better: 
%
%{\small\verb!\hypersetup{hidelinks}!}.
%
%\noindent If you want to have obvious links in the PDF but not the printed text, use:
%
%{\small\verb!\hypersetup{colorlinks=false}!}.
