%%%%%%%%%%%%%%%%%%%%%%%%%%%%%%%%%%%%%%%%%
% Masters/Doctoral Thesis 
% LaTeX Template
% Version 2.5 (27/8/17)
%
% This template was downloaded from:
% http://www.LaTeXTemplates.com
%
% Version 2.x major modifications by:
% Vel (vel@latextemplates.com)
%
% This template is based on a template by:
% Steve Gunn (http://users.ecs.soton.ac.uk/srg/softwaretools/document/templates/)
% Sunil Patel (http://www.sunilpatel.co.uk/thesis-template/)
%
% Template license:
% CC BY-NC-SA 3.0 (http://creativecommons.org/licenses/by-nc-sa/3.0/)
%
%%%%%%%%%%%%%%%%%%%%%%%%%%%%%%%%%%%%%%%%%

%----------------------------------------------------------------------------------------
%	PACKAGES AND OTHER DOCUMENT CONFIGURATIONS
%----------------------------------------------------------------------------------------

\documentclass[
11pt, % The default document font size, options: 10pt, 11pt, 12pt
oneside, % Two side (alternating margins) for binding by default, uncomment to switch to one side
english, % ngerman for German
%singlespacing, % Single line spacing, alternatives: onehalfspacing or doublespacing
doublespacing,
%draft, % Uncomment to enable draft mode (no pictures, no links, overfull hboxes indicated)
%nolistspacing, % If the document is onehalfspacing or doublespacing, uncomment this to set spacing in lists to single
%liststotoc, % Uncomment to add the list of figures/tables/etc to the table of contents
%toctotoc, % Uncomment to add the main table of contents to the table of contents
%parskip, % Uncomment to add space between paragraphs
%nohyperref, % Uncomment to not load the hyperref package
headsepline, % Uncomment to get a line under the header
%chapterinoneline, % Uncomment to place the chapter title next to the number on one line
%consistentlayout, % Uncomment to change the layout of the declaration, abstract and acknowledgements pages to match the default layout
]{MastersDoctoralThesis} % The class file specifying the document structure

\usepackage[utf8]{inputenc} % Required for inputting international characters
\usepackage[T1]{fontenc} % Output font encoding for international characters

\usepackage{mathpazo} % Use the Palatino font by default


\usepackage[backend=bibtex,style=authoryear,natbib=true]{biblatex} % Use the bibtex backend with the authoryear citation style (which resembles APA)

\addbibresource{example.bib} % The filename of the bibliography

\usepackage[autostyle=true]{csquotes} % Required to generate language-dependent quotes in the bibliography


%----------------------------------------------------------------------------------------
%% include hebrew letters

%\usepackage{ucs}   % package to add unicode support
%\usepackage[utf8x]{inputenc}  % adding the UTF-8 encoding
%\usepackage[english,hebrew]{babel}  % telling babel: english & hebrew in doc.
%\usepackage{hebfont}  % Adding a selection of fonts.

%\usepackage[hebrew,english]{babel}
%\usepackage{hyperref}
%\HeblatexRedefineL  % this stands for \def\L{\protect\pL}

%\textbf{\usepackage{amsmath,amssymb}
%
%\DeclareFontFamily{U}{rcjhbltx}{}
%\DeclareFontShape{U}{rcjhbltx}{m}{n}{<->rcjhbltx}{}
%\DeclareSymbolFont{hebrewletters}{U}{rcjhbltx}{m}{n}
%
%% remove the definitions from amssymb
%\let\aleph\relax\let\beth\relax
%\let\gimel\relax\let\daleth\relax
%
%\DeclareMathSymbol{\aleph}{\mathord}{hebrewletters}{39}
%\DeclareMathSymbol{\beth}{\mathord}{hebrewletters}{98}\let\bet\beth
%\DeclareMathSymbol{\gimel}{\mathord}{hebrewletters}{103}
%\DeclareMathSymbol{\daleth}{\mathord}{hebrewletters}{100}\let\dalet\daleth
%
%\DeclareMathSymbol{\lamed}{\mathord}{hebrewletters}{108}
%\DeclareMathSymbol{\mem}{\mathord}{hebrewletters}{109}\let\mim\mem
%\DeclareMathSymbol{\ayin}{\mathord}{hebrewletters}{96}
%\DeclareMathSymbol{\tsadi}{\mathord}{hebrewletters}{118}
%\DeclareMathSymbol{\qof}{\mathord}{hebrewletters}{114}
%\DeclareMathSymbol{\shin}{\mathord}{hebrewletters}{152}
% \usepackage{cjhebrew}
%%----------------------------------------------------------------------------------------

%----------------------------------------------------------------------------------------
%	MARGIN SETTINGS
%----------------------------------------------------------------------------------------

\geometry{
	paper=a4paper, % Change to letterpaper for US letter
	inner=2.5cm, % Inner margin
	outer=3.8cm, % Outer margin
	bindingoffset=.5cm, % Binding offset
	top=1.5cm, % Top margin
	bottom=1.5cm, % Bottom margin
	%showframe, % Uncomment to show how the type block is set on the page
}

%----------------------------------------------------------------------------------------
%	THESIS INFORMATION
%----------------------------------------------------------------------------------------

%\thesistitle{Computing angles of plant root tips} % Your thesis title, this is used in the title and abstract, print it elsewhere with \ttitle
%\thesistitle{Software tool to compute angles of plant root tips}
\thesistitle{\textit{RootSkel} -- A software tool to measure curved plant root tips}
\supervisor{Dr Giovanni \textsc{Sena}} % Your supervisor's name, this is used in the title page, print it elsewhere with \supname
\examiner{} % Your examiner's name, this is not currently used anywhere in the template, print it elsewhere with \examname
\degree{MSc Bioinformatics and Theoretical Systems Biology} % Your degree name, this is used in the title page and abstract, print it elsewhere with \degreename
\author{Felicia \textsc{Burtscher}} % Your name, this is used in the title page and abstract, print it elsewhere with \authorname
\addresses{} % Your address, this is not currently used anywhere in the template, print it elsewhere with \addressname

%\subject{Biological Sciences} % Your subject area, this is not currently used anywhere in the template, print it elsewhere with \subjectname
\keywords{} % Keywords for your thesis, this is not currently used anywhere in the template, print it elsewhere with \keywordnames
\university{\href{http://www.university.com}{Imperial College London}} % Your university's name and URL, this is used in the title page and abstract, print it elsewhere with \univname
\department{\href{http://department.university.com}{Department of Life Sciences}} % Your department's name and URL, this is used in the title page and abstract, print it elsewhere with \deptname
%\wordcount{Word count: INSERT WORDS HERE}
\group{\href{http://researchgroup.university.com}{Laboratory of Plant Morphogenesis}} % Your research group's name and URL, this is used in the title page, print it elsewhere with \groupname
\faculty{}
%\faculty{\href{http://faculty.university.com}{Department of Life Sciences}} % Your faculty's name and URL, this is used in the title page and abstract, print it elsewhere with \facname

\AtBeginDocument{
\hypersetup{pdftitle=\ttitle} % Set the PDF's title to your title
\hypersetup{pdfauthor=\authorname} % Set the PDF's author to your name
\hypersetup{pdfkeywords=\keywordnames} % Set the PDF's keywords to your keywords
}

\begin{document}

\frontmatter % Use roman page numbering style (i, ii, iii, iv...) for the pre-content pages

\pagestyle{plain} % Default to the plain heading style until the thesis style is called for the body content

%----------------------------------------------------------------------------------------
%	TITLE PAGE
%----------------------------------------------------------------------------------------

\begin{titlepage}
\begin{center}

\vspace*{.06\textheight}
{\scshape\LARGE \univname\par}\vspace{1.5cm} % University name
\textsc{\Large Report of project 3}\\[0.5cm] % Thesis type

\HRule \\[0.4cm] % Horizontal line
{\huge \bfseries \ttitle\par}\vspace{0.4cm} % Thesis title
\HRule \\[1.5cm] % Horizontal line
 
\begin{minipage}[t]{0.4\textwidth}
\begin{flushleft} \large
\emph{Author:}\\
\href{http://www.johnsmith.com}{\authorname} % Author name - remove the \href bracket to remove the link
\end{flushleft}
\end{minipage}
\begin{minipage}[t]{0.4\textwidth}
\begin{flushright} \large
\emph{Supervisor:} \\
\href{http://www.jamessmith.com}{\supname} % Supervisor name - remove the \href bracket to remove the link  
\end{flushright}
\end{minipage}\\[3cm]
 
\vfill

\large \textit{A thesis submitted in fulfillment of the requirements\\ for the degree of \degreename}\\[0.3cm] % University requirement text
\textit{in the}\\[0.4cm]
%\groupname\\\deptname\\[2cm] % Research group name and department name
 \deptname\\[1cm]
\vfill

{\large \today}\\[2cm] % Date
%\includegraphics{Logo} % University/department logo - uncomment to place it
 
 {Word count: INSERT \#WORDS HERE}\\[2cm] % Date
 
\vfill
\end{center}
\end{titlepage}

 %----------------------------------------------------------------------------------------
 %	DECLARATION PAGE
 %----------------------------------------------------------------------------------------

% \begin{declaration}
% \addchaptertocentry{\authorshipname} % Add the declaration to the table of contents
% \noindent I, \authorname, declare that this thesis titled, \enquote{\ttitle} and the work presented in it are my own. I confirm that:
% \begin{itemize} 
% \item This work was done wholly or mainly while in candidature for a research degree at this University.
% \item Where any part of this thesis has previously been submitted for a degree or any other qualification at this University or any other institution, this has been clearly stated.
% \item Where I have consulted the published work of others, this is always clearly attributed.
% \item Where I have quoted from the work of others, the source is always given. With the exception of such quotations, this thesis is entirely my own work.
% \item I have acknowledged all main sources of help.
% \item Where the thesis is based on work done by myself jointly with others, I have made clear exactly what was done by others and what I have contributed myself.\\
% \end{itemize}
% \noindent Signed:\\
% \rule[0.5em]{25em}{0.5pt} % This prints a line for the signature
% \noindent Date:\\
% \rule[0.5em]{25em}{0.5pt} % This prints a line to write the date
% \end{declaration}

% \cleardoublepage

%----------------------------------------------------------------------------------------
%	QUOTATION PAGE
%----------------------------------------------------------------------------------------

\vspace*{0.2\textheight}

%\noindent\enquote{\itshape Thanks to my solid academic training, today I can write hundreds of words on virtually any topic without possessing a shred of information, which is how I got a good job in journalism.}\bigbreak
%
%\hfill Dave Barry

\noindent\enquote{\itshape If you thought that science was certain -- well, that is just an error on your part.}\bigbreak

\hfill Richard Feynman



%----------------------------------------------------------------------------------------
%	ABSTRACT PAGE
%----------------------------------------------------------------------------------------

\begin{abstract}
\addchaptertocentry{\abstractname} % Add the abstract to the table of contents
% The Thesis Abstract is written here (and usually kept to just this page). The page is kept centered vertically so can expand into the blank space above the title too\ldots

% OVERVIEW / STAND-ALONE SUMMARY OF THE WORK 
% < 300 words
% summary that identifies the purpose, problem, methods, results, and conclusion of your work
% In the following work we we developed a root tip plant\\
%  1) the overall purpose of the study and the research problem(s) you investigated; \\
%  2) the basic design of the study; \\
%  3) major findings or trends found as a result of your analysis; and, \\
%  4) a brief summary of your interpretations and conclusions.

Plant morphology studies the form and structure of plants and how these develop over time under different circumstances. \\
%It compares and interprets these structures in various plants of the same or different species which can lead to insights into the underlying biology, and the understanding of plant relationships and plant evolution.\\
One phenomenon that, despite being first identified more than 100 years ago, has not been researched much and we know little about its underlying molecular mechanisms, is electrotropism.
Electrotropism describes the response of a plant to an electric field; one common response is the curving of a plant’s root tip. Unlike gravitropism [DO I HAVE TO EXPLAIN IT HERE?] there exist to our knowledge no tools specifically designed to assist biologists in studying this effect, most importantly the angle resulting from the curving of root tips. This is especially relevant as the experiment setup is more complex and the images tend to be more error-prone, which is often the reason why standard gravitropism study tools fail and biologists compute angle resulting from the curved root tip manually.

Here we present \textit{RootSkel}, a novel intuitive and robust stand-alone software for image processing developed in \textit{MATLAB}, whose pipeline was optimised for noise-intensive electrotropism images. % built from the ground up. 
Unlike when doing the angle computation to measure the curvature of a root tip manually, our tool ensures a standardised version of the angle computation. To make the tool more user-friendly, we developed a graphical user interface (GUI) that will help the user in processing the images and computing the angles in a standardised and controlled fashion.

%Our data set consisted of high-throughput time-lapse images of Arabidopsis roots taken by a standard Raspberry Pi V2 camera from 5 experiments with 5-6 roots each containing between 32 and 36 images over a period of approximately 5 hours. 

% produced by the Plant Morphology lab at Imperial College London. 
%We will present an example for the user on how to use the tool.

%Additionally, we evaluated our results by comparing it to previously manually computed angles. [INCLUDE RESULTS A LA 70\% of the images could be found to be reproduced]

Additionally we computed the angle of three Arabidopsis root tips at 30 time points over one whole experiment (approximately 5 hours) and evaluated the results by comparing it to previously manually computed angles on the same images. The results show same patterns as manual computed angles;
% overall correspondence with manual angles; 
statistics on our data set suggest that our tool removes human error in the angle calculation process %which accounts in average [INSERT VARIANCE HERE, STATE HOW MUCH HUMAN VARIANCE WE COULD ELIMINATE BY STANDARDISING IT] 
of up to 10\% of the actual angle. However, further validation on more images need to be done. Unlike the manually computed angles, with using a standard definition of the angle, our software delivers angles free from human bias which makes the results comparable and reproducible. 
% Indeed, we show that on this example we can eliminate in average [INSERT VARIANCE HERE, STATE HOW MUCH HUMAN VARIANCE WE COULD ELIMINATE BY STANDARDISING IT] of human bias; further validation on more images is to be done.
%can be done on already existing data as well as data gathered in the future. 
Previously computed angles can be checked by using our software, and more angles of curved roots can be computed in the future and hopefully %help 
reveal interesting insights in electrotropism. 
Moreover, our tool is not limited to roots but could theoretically be used on any curved object; slight modifications in the code and the GUI might be necessary.


%AIMS
%METHODS
%RESULTS 
%CONCLUSION
%
%CITE SOME KEY NUMERICAL RESULTS RATHER THAN JUST GENERALITIES
 
\end{abstract}

%----------------------------------------------------------------------------------------
%	ACKNOWLEDGEMENTS
%----------------------------------------------------------------------------------------

\begin{acknowledgements}
\addchaptertocentry{\acknowledgementname} % Add the acknowledgements to the table of contents

This project was conducted under the supervision of Dr Giovanni Sena, whom I thank for his advice and guidance. Thanks also to Nick Oliver for their help and expertise from the biological side and other members from the Plant Morphogenesis Laboratory at Imperial College London, as well as Suhail A Islam for fixing technical issues and his incredible patience. Finally, thanks to Prof Michael PH Stumpf, Prof Michael Sternberg and others involved in conducting and overseeing the MSc in Bioinformatics and Theoretical Systems Biology at Imperial College London.

%\ldots
\end{acknowledgements}


%----------------------------------------------------------------------------------------
%	LIST OF CONTENTS/FIGURES/TABLES PAGES
%----------------------------------------------------------------------------------------

\tableofcontents % Prints the main table of contents

\listoffigures % Prints the list of figures

\listoftables % Prints the list of tables

%----------------------------------------------------------------------------------------
%	ABBREVIATIONS
%----------------------------------------------------------------------------------------

\begin{abbreviations}{lll} % Include a list of abbreviations (a table of two columns)

%\textbf{AORC} & \textbf{A}node-\textbf{o}riented \textbf{R}oot \textbf{C}urvature & \\
%\textbf{CORC} & \textbf{C}athode-\textbf{o}riented \textbf{R}oot \textbf{C}urvature & \\
\textbf{EF} & \textbf{E}lectric \textbf{F}ield & \\
\textbf{GUI} & \textbf{G}raphical \textbf{U}ser \textbf{I}nterface & \\
\textbf{ie} & \textit{latin} \textbf{i}d \textbf{e}st & that is\\

%\textbf{LAH} & \textbf{L}ist \textbf{A}bbreviations \textbf{H}ere & \\
%\textbf{WSF} & \textbf{W}hat (it) \textbf{S}tands \textbf{F}or & \\

\end{abbreviations}

% %----------------------------------------------------------------------------------------
% %	PHYSICAL CONSTANTS/OTHER DEFINITIONS
% %----------------------------------------------------------------------------------------

% \begin{constants}{lr@{${}={}$}l} % The list of physical constants is a three column table

% % The \SI{}{} command is provided by the siunitx package, see its documentation for instructions on how to use it

% Speed of Light & $c_{0}$ & \SI{2.99792458e8}{\meter\per\second} (exact)\\
% %Constant Name & $Symbol$ & $Constant Value$ with units\\

% \end{constants}

% %----------------------------------------------------------------------------------------
% %	SYMBOLS
% %----------------------------------------------------------------------------------------

% \begin{symbols}{lll} % Include a list of Symbols (a three column table)

% $a$ & distance & \si{\meter} \\
% $P$ & power & \si{\watt} (\si{\joule\per\second}) \\
% %Symbol & Name & Unit \\

% \addlinespace % Gap to separate the Roman symbols from the Greek

% $\omega$ & angular frequency & \si{\radian} \\

% \end{symbols}

 %----------------------------------------------------------------------------------------
 %	DEDICATION
 %----------------------------------------------------------------------------------------

 \dedicatory{For Yaron Efrat -- the person I admire the most.} 
 
 %\textaleph \hebnun \hebyod 
 %
 %\selectlanguage{hebrew}
 
 %	אני אוהבת אותך 
 % 	%מתוק
 % 	נפש תאומה שלי}
 % תודה על היותם בשבילי
 
 
 

%----------------------------------------------------------------------------------------
%	THESIS CONTENT - CHAPTERS
%----------------------------------------------------------------------------------------

\mainmatter % Begin numeric (1,2,3...) page numbering

\pagestyle{thesis} % Return the page headers back to the "thesis" style

% Include the chapters of the thesis as separate files from the Chapters folder
% Uncomment the lines as you write the chapters

% Chapter 1: INTRODUCTION

\chapter{Introduction} % Main chapter title

\label{introduction} % For referencing the chapter elsewhere, use \ref{Chapter1} 

%----------------------------------------------------------------------------------------

% Define some commands to keep the formatting separated from the content 
\newcommand{\keyword}[1]{\textbf{#1}}
\newcommand{\tabhead}[1]{\textbf{#1}}
\newcommand{\code}[1]{\texttt{#1}}
\newcommand{\file}[1]{\texttt{\bfseries#1}}
\newcommand{\option}[1]{\texttt{\itshape#1}}

%----------------------------------------------------------------------------------------

RELEVANCE AND TOPICALITY OF THE AIMS AND OBJECTIVES AND MY CONTRIBUTION TO THE RESEARCH.
HOW HAS THE PROJECT ADVANCED THE FIELD?

%----------------------------------------------------------------------------------------

\section{Biological background}

An important biological phenomenon studied by plant morphologists is \textit{tropism} [ADD TO GLOSSARY], which is used to indicate the turning movement of a biological organism, here of a plant, when exposed to different environmental simuli [INSERT REFERENCE]. Usually the stimulus involved is added to the name, eg \textit{phototropism} as a reaction to sunlight; it can be either \textit{positive}, ie towards the stimulus, or \textit{negative}, ie away from the stimulus.
The most frequently observed and best studied tropism in \textit{gravitropism}, which describes the process of how plants grow as a response to gravity. It was firstly scientifically documented by Charles Darwin [INSERT REFERENCE] and can be observed in higher and many lower plants as well as other organisms [INSERT REFERENCE]: Roots show \textit{positive gravitropism}, ie they grow in the direction of the gravitational pull whereas stems grow in the opposite direction. An easy experiment to do is to lay a potted plant onto its side; over time the stem will begin to turn upwards and thus show negative gravitropism. \\
A far less studied process is \textit{electrotropism} which describes the growth or movement of a plant when exposed to an electric field and which was the tropism under study in this project. 

A high-overview explanation of the experiment setup and data collection can be found in section [INSERT REFERENCE TO SECTION METHOD].

%----------------------------------------------------------------------------------------

\section{Literature review}

\subsection{RootTrace}

RootTrace [INSERT REFERENCE] has been developed to measure root lengths across time serie image data; biologists have also used it to measuer highly curved roots. A graphical user interface (GUI) implemented within the RootTrace framework makes it easy for the user to handle. However, this tool has failed on our image data set [INSERT REFERENCE HERE], probably due to the high noise level found in electrotropism images compared to gravitropism images.






\subsection{PlantCV}


COULD HAVE USED ONE OF THESE APPRAOCHES, BUT STARTED FROM SCRATCH. BASED ON THIS DATA SET.


However, 


Electopism much much harder since photos noisier.



%%----------------------------------------------------------------------------------------
%%----------------------------------------------------------------------------------------
%%----------------------------------------------------------------------------------------
%
%\section{What this Template Includes}
%
%\subsection{Folders}
%
%\subsection{Files}
%
%\keyword{main.out} -- this is an auxiliary file generated by \LaTeX{}, if it is deleted \LaTeX{} simply regenerates it when you run the main \file{.tex} file.
%
%
%%----------------------------------------------------------------------------------------
%
%\section{Filling in Your Information in the \file{main.tex} File}\label{FillingFile}
%
%
%
%%----------------------------------------------------------------------------------------
%
%\section{The \code{main.tex} File Explained}
%
%
%
%%----------------------------------------------------------------------------------------
%
%\section{Thesis Features and Conventions}\label{ThesisConventions}
%
%
%\subsection{Printing Format}
%
%This thesis template is designed for double sided printing (i.e. content on the front and back of pages) as most theses are printed and bound this way. Switching to one sided printing is as simple as uncommenting the \option{oneside} option of the \code{documentclass} command at the top of the \file{main.tex} file. You may then wish to adjust the margins to suit specifications from your institution.
%
%The headers for the pages contain the page number on the outer side (so it is easy to flick through to the page you want) and the chapter name on the inner side.
%
%The text is set to 11 point by default with single line spacing, again, you can tune the text size and spacing should you want or need to using the options at the very start of \file{main.tex}. The spacing can be changed similarly by replacing the \option{singlespacing} with \option{onehalfspacing} or \option{doublespacing}.
%
%
%\subsection{References}
%
%The \code{biblatex} package is used to format the bibliography and inserts references such as this one \parencite{Reference1}. The options used in the \file{main.tex} file mean that the in-text citations of references are formatted with the author(s) listed with the date of the publication. Multiple references are separated by semicolons (e.g. \parencite{Reference2, Reference1}) and references with more than three authors only show the first author with \emph{et al.} indicating there are more authors (e.g. \parencite{Reference3}). This is done automatically for you. To see how you use references, have a look at the \file{Chapter1.tex} source file. Many reference managers allow you to simply drag the reference into the document as you type.
%
%Scientific references should come \emph{before} the punctuation mark if there is one (such as a comma or period). The same goes for footnotes\footnote{Such as this footnote, here down at the bottom of the page.}. You can change this but the most important thing is to keep the convention consistent throughout the thesis. Footnotes themselves should be full, descriptive sentences (beginning with a capital letter and ending with a full stop). The APA6 states: \enquote{Footnote numbers should be superscripted, [...], following any punctuation mark except a dash.} The Chicago manual of style states: \enquote{A note number should be placed at the end of a sentence or clause. The number follows any punctuation mark except the dash, which it precedes. It follows a closing parenthesis.}
%
%The bibliography is typeset with references listed in alphabetical order by the first author's last name. This is similar to the APA referencing style. To see how \LaTeX{} typesets the bibliography, have a look at the very end of this document (or just click on the reference number links in in-text citations).
%
%\subsubsection{A Note on bibtex}
%
%The bibtex backend used in the template by default does not correctly handle unicode character encoding (i.e. "international" characters). You may see a warning about this in the compilation log and, if your references contain unicode characters, they may not show up correctly or at all. The solution to this is to use the biber backend instead of the outdated bibtex backend. This is done by finding this in \file{main.tex}: \option{backend=bibtex} and changing it to \option{backend=biber}. You will then need to delete all auxiliary BibTeX files and navigate to the template directory in your terminal (command prompt). Once there, simply type \code{biber main} and biber will compile your bibliography. You can then compile \file{main.tex} as normal and your bibliography will be updated. An alternative is to set up your LaTeX editor to compile with biber instead of bibtex, see \href{http://tex.stackexchange.com/questions/154751/biblatex-with-biber-configuring-my-editor-to-avoid-undefined-citations/}{here} for how to do this for various editors.
%
%
%\begin{table}
%\caption{The effects of treatments X and Y on the four groups studied.}
%\label{tab:treatments}
%\centering
%\begin{tabular}{l l l}
%\toprule
%\tabhead{Groups} & \tabhead{Treatment X} & \tabhead{Treatment Y} \\
%\midrule
%1 & 0.2 & 0.8\\
%2 & 0.17 & 0.7\\
%3 & 0.24 & 0.75\\
%4 & 0.68 & 0.3\\
%\bottomrule\\
%\end{tabular}
%\end{table}
%
%You can reference tables with \verb|\ref{<label>}| where the label is defined within the table environment. See \file{Chapter1.tex} for an example of the label and citation (e.g. Table~\ref{tab:treatments}).
%
%\subsection{Figures}
%
%There will hopefully be many figures in your thesis (that should be placed in the \emph{Figures} folder). The way to insert figures into your thesis is to use a code template like this:
%\begin{verbatim}
%\begin{figure}
%\centering
%\includegraphics{Figures/Electron}
%\decoRule
%\caption[An Electron]{An electron (artist's impression).}
%\label{fig:Electron}
%\end{figure}
%\end{verbatim}
%Also look in the source file. Putting this code into the source file produces the picture of the electron that you can see in the figure below.
%
%\begin{figure}[th]
%\centering
%\includegraphics{Figures/Electron}
%\decoRule
%\caption[An Electron]{An electron (artist's impression).}
%\label{fig:Electron}
%\end{figure}
%
%Sometimes figures don't always appear where you write them in the source. The placement depends on how much space there is on the page for the figure. Sometimes there is not enough room to fit a figure directly where it should go (in relation to the text) and so \LaTeX{} puts it at the top of the next page. Positioning figures is the job of \LaTeX{} and so you should only worry about making them look good!
%
%Figures usually should have captions just in case you need to refer to them (such as in Figure~\ref{fig:Electron}). The \verb|\caption| command contains two parts, the first part, inside the square brackets is the title that will appear in the \emph{List of Figures}, and so should be short. The second part in the curly brackets should contain the longer and more descriptive caption text.
%
%The \verb|\decoRule| command is optional and simply puts an aesthetic horizontal line below the image. If you do this for one image, do it for all of them.
%
%\LaTeX{} is capable of using images in pdf, jpg and png format.
%
%
%%----------------------------------------------------------------------------------------
%
%\begin{flushright}
%Guide written by ---\\
%Sunil Patel: \href{http://www.sunilpatel.co.uk}{www.sunilpatel.co.uk}\\
%Vel: \href{http://www.LaTeXTemplates.com}{LaTeXTemplates.com}
%\end{flushright}

% Chapter 2: METHODS

\chapter{Methods} % Main chapter title

\label{methods} % For referencing the chapter elsewhere, use \ref{Chapter1} 

%%----------------------------------------------------------------------------------------
%
%% Define some commands to keep the formatting separated from the content 
%\newcommand{\keyword}[1]{\textbf{#1}}
%\newcommand{\tabhead}[1]{\textbf{#1}}
%\newcommand{\code}[1]{\texttt{#1}}
%\newcommand{\file}[1]{\texttt{\bfseries#1}}
%\newcommand{\option}[1]{\texttt{\itshape#1}}

%----------------------------------------------------------------------------------------

\section{Image processing}

Image processing pools together a lot of different domains including physics (optics), signal processing and pattern recognition/ Machine Learning (ML) to ultimately feed computer to understand how do interpret images and make decisions based on them. 

combines optics and signal processing and is often used in computer vision.
\begin{enumerate}
	\item Image aquisition
	\item Image preprocessing 
	\item Image segmentation
	\item Image representation and description -- dep on image that you are studying
	\item Image understanding
	\item Results (output)
\end{enumerate}

\subsection{What is a digital image?}

We operate on digital (discrete) images:
\begin{itemize}
	\item Sample the 2D space on a regular grid
	\item Quantise each sample, ie round to nearest integer
\end{itemize}

If our samples are \( \Delta \) apart, we can write this as:
\[
f(x,y) = Quantize{f(\Delta x, \Delta y)}
\]
The image can now be represented as a matrix of integer values:

[INSERT EXAMPLE PICTURE HERE]


\subsection{Image processing operation}

An image processing operation typically defines a new image \( g \) in terms of an existing image \( f \).
\begin{itemize}
	\item Transform the range of \( f \)
	\[ 
	g(x, y) = t(f(x,y)) 
	\]
	\item Transform the domain of \( f \)
	\[
	g(x,y) = f(t_{x}(x,y), t_{y}(x,y))
	\]
\end{itemize}

\subsection{Key stages in digital image processing}

[INSERT FLOW CHART HERE]

	Problem Domain -- here plant morphology 
\begin{itemize}
	\item Image Aquisition
	\item Image Enhancement -- contrasting images so human eye can best see things, manipulate values of pixels so you can best caracterise and structure them |
	\item Morphological Processing -- growing and thinning of pixels, eg fingerpringt medical operations
	\item Image Segmentation -- separating parts of the image such as features and object you want to look at, distribution of pixel values |
	\item Representation \& Description -- based on quantised space that you're working in. What's realised in that digital output? Two ways to describe features being observed in that pixel space. Inverting etc
	\item Object Recognition -- see example iris data set. cluster to identify different objects.
	\item [Computer \& Machine Vision] -- feeds into computer vision and ML
\end{itemize}
	[Robotics \& AI], [Deep Learning]
	
	feature space that you're working in on top of your labels.
	put in our own bias and how we train model.
	
	A lot of noise in the image makes it grainy. We can apply interpolation, ie smoothing the image based on its nn. bi-cubic interpolation.  Smoothes some of the details as well as a side effect, but especially the noise. -- important for edge detection later

%----------------------------------------------------------------------------------------

\section{Matlab}

As a language we chose Matlab as it has a well-documented image processing toolbox with a very good documentation. Alternative languages are Python and Julia. 

ImageJ

Avizo

 
%----------------------------------------------------------------------------------------

\section{Workflow}


Generic image processing workflow:

However, every single case is different. 

Skeletonisation

fill in holes


[INSERT FLOWCHART WITH SINGLE STEPS]

%0) Convert the image from rgb to gray scale (matlab: rgb2gray) (3-tensor to matrix)
%
%intensify the matrix \arrowvert distinguish roots for user
%highlight window \arrowvert help user where approximately roots are

%
%1. The user will select a region of interest on the image so only roots appear. This window can be used over the whole experiment (as camera doesn’t move) ///TODO: or better to define window for each root?
%
%2. The user then clicks on some points over the roots (I'd say at least 7, preferably 10). Using the mean value and standard deviation of the pixel intensity of these points a thresholding range is found. Say [mu-2sigma,mu+2sigma]. 
%Initially trying to use the same threshold for all the images in the experiment, maybe this will do the job already…
%
%3. After segmenting the images, the binary images need to postprocessed so we get rid of the bright spots and noise, eg by removing any object that has fewer than 1000 pixels (matlab: bwareaopen enough?). Then any holes that might appear in the roots need to be filled (matlab: imfill)

%
%4. Hopefully obtaining the skeleton version of the binary image (matlab: bwmorph).
%
%5. Label each individual object in the image so they can later be processed independently (matlab: bwconncomp).

%6. For each root we then get the list of point belonging to the tip (upper x- and y-values) and we compute the curvature (eg on a 3-point polygonal chain)

\section{Pre-processing -- getting the skeleton}

The initial problem of coming up with a "good" definition by comparing different angle and curvature definition shifted. The focus was now on the pre-processing, the extracting the skeleton. 
However, we will still present some possible definitions of angles and curvature, also not implemented in the current version of the tool.

\section{Computing the angle}

Once you defined the lines, it is straight-forward to compute the angle.

Eg using Hough transform from Matlab Image Processing Toolbox to detect lines in an image not applicable here as highly pixled image.

We refer to figure [INSERT REFERENCE HERE] to how the angle was computed in previous approaches. 

However, in the software described here the main goal was to emulate the angle that has so far been computed manually. 
Approaches like computing the Gaussian mean curvature, definitions of different, possibly more robust methods of computing and angle are mentioned here, but has not been incorporated in our final tool. 

\subsection{Using the position of the tip of the root -- more in detail}

INSERT FIGURE OF WHAT ANGLE WE CALCULATE AND WHAT IS THE ANGLE OF INTEREST. 
USER-FRIENDLY: EASILY SWITCH BETWEEN THE TWO OF THEM.

We find the angle between the two lines (point of highest mean curvture OR manually click point of highest curvature and tip and point of highest mean curvature and equally distant point/ pixel on root).

There are other ways such as finding vectors on the lines and using their dot products. However, we can use simple, basic trigonometric methods as shown in figure ....

We would then compute the angles of the two lines to the x-axis  in radians mode by 
\[
\pi - | \tan^{-1}(\frac{x(2) - x(1)}{y(2) - y(1)}) - \tan^{-1}(\frac{z(2) - z(1)}{y(2) - y(1)}) |
\]
or 
\[
180^{\circ} - | \tan^{-1}(\frac{x(2) - x(1)}{y(2) - y(1)}) - \tan^{-1}(\frac{z(2) - z(1)}{y(2) - y(1)}) |
\]
if we wanted it in degrees mode.

The single steps are shown in figure ....
[PSEUDOCODE]
\begin{enumerate}
	\item Find the slope of each line.
	\item Find the inclination of each line using \( \alpha = \tan^{-1} m \), where  \( \alpha \) is the angle of inclination, \( m \) is the slope.
	\item Take the difference of these two angles.
	\item Handle the case where this difference is not an acute, ie less than 90 degrees, angle. If we get a negative angle, we take its absolute value. We want to find an acute angle, so if we calculated an obtuse angle, oe greater than 90 degrees, we just subtract the value from pi radians or 180 degrees to get the acute values.
\end{enumerate}


\subsection{Using the angle of the tangents to the points that are x points/pixels away from the point of highest local curvature}

\subsection{INSERT VARIOUS OTHER IDEAS OF ANGLE COMPUTATION -- even if not implemented}
 
% Chapter 3: RESULTS

\chapter{Results} % Main chapter title

\label{results} % For referencing the chapter elsewhere, use \ref{Chapter1} 

%%----------------------------------------------------------------------------------------
%
%% Define some commands to keep the formatting separated from the content 
%\newcommand{\keyword}[1]{\textbf{#1}}
%\newcommand{\tabhead}[1]{\textbf{#1}}
%\newcommand{\code}[1]{\texttt{#1}}
%\newcommand{\file}[1]{\texttt{\bfseries#1}}
%\newcommand{\option}[1]{\texttt{\itshape#1}}

%----------------------------------------------------------------------------------------

%PROVIDE A HIGH-LEVEL REVIEW, NOT LOTS OF NUMBERS

%REPRODUCABILITY OF WORK, ie make sure reader could redo it

%%----------------------------------------------------------------------------------------
%%----------------------------------------------------------------------------------------

The following section of this report will briefly explain the tool from a technical perspective, but more importantly  %and show how to use it in practice. %Additionally, we present an example of the tool in use.
we will present some highlights of the tool that sets it apart from other tools. In the appendix [INSERT REFERENCE HERE] we will guide the user through one example to illustrate how the tool is used in practice. Tthis can serve as a manual for users even though the steps are self-explanatory by just using the GUI. 

Additionally, we will show some validation of the tool by comparing the angle computed by our tool to the one computed manually on 3 time-series image data set of one Arabidopsis roots. 

The code is open-source and publicly available on [INSERT GITHUB REFERENCE HERE]; all previous versions including log files can be found on [INSERT GITHUB REFERENCE HERE].
%EMPHASISE THIS: MAKE RESEARCH TRANSPARENT


%----------------------------------------------------------------------------------------
%----------------------------------------------------------------------------------------
\section{Key features} %SHOW HOW ELEABORATE TOOL IS

Figure [INSERT REFERENCE HERE: PIPELINE FROM GUI] explains the key components of this image-analysis software tool to address the problem of highly noisy electrotropism consumer camera images of Arabidopsis roots and a standardised way of computing the angle for the curved root tip. 

This tool takes the form of a MATLAB program and subprograms with a graphical user interface on top of it.

%----------------------------------------------------------------------------------------
\subsection{Graphical User Interface}
To make the program more user-friendly, we developed a graphical user interface (GUI). Pop-up windows, message boxes as well as error boxes, will guide the user through the process (see [INSERT REFERENCE HERE]); a separate manual is not necessary as the steps are very intuitive and straight-forward.
There are various benefits of the GUI such as
\begin{itemize}
	\item Visualising the process including the pipeline and the angle that is computed
	\item Easy user interaction with mouse clicking
	\item Flexibility, eg the user can go back at each step without running the whole script from the beginning
	\item Pop-up windows and mouseover functions on buttons that guide the user through the process and explain the steps
	\item Error messages if user does not enter allowed values.
\end{itemize}


%%----------------------------------------------------------------------------------------
%\subsection{Discerning root from background}
%
%Gamma correction (imadjust)
%[TO COMPLETE]


%----------------------------------------------------------------------------------------
\subsection{Handling user's mistakes}

When we take the user's input, eg choosing samples along the root, we correct for small mistakes by taking a neighbourhood (3 \(\times\)3) average around the pixel. 
%followed by adaptive thresholding.
This means the user does not have to take special care when choosing the points as long as it is in the approximate region of the root.

%----------------------------------------------------------------------------------------
\subsection{User interaction and optional steps}
The software tool was created in ways that it is easy to interact with for a future user. 
We implemented several optional steps that only need to be performed if the user thinks they are necessary. This on the other hand saves time in the preprocessing but on the other hand also ensures that tricky roots can be tackled by various optional steps in order to extract a skeleton. 

%----------------------------------------------------------------------------------------
\subsection{Drawing the angle}
The GUI lets the user visualise the angle that is computed. This not only helps to make the tool more visual and transparent, but can also assist in debugging. 


%----------------------------------------------------------------------------------------
%----------------------------------------------------------------------------------------
\section{Workflow of root skeletonisation -- explained on GUI}

The pipeline that has been developed for the preprocessing step of extracting the skeleton is displayed in figure [INSERT REFERENCE HERE] and can be viewed at the bottom of the GUI, see figure [INSERT REFERENCE HERE].
%In the appendix [INSERT REFERENCE HERE] we present the tool on one example image guides the reader through the pipeline and can serve as a manual for future users.

%[INSERT FIGURE OF DIFFERENT STEPS (SUBFIGURES) THAT ARE THEN REFERRED TO IN THE TEXT INDIVIDUALLY]

\begin{figure}[h]
	\centering
	\includegraphics[width=0.6\textwidth]{../Figures/workflow_big.png}
	\caption{The workflow of \textit{RootSkel} and different component of the GUI: Pre-processing steps to extract the skeleton are highlighted in orange, the actual angle computation is shaded in blue. Steps that have a front-end, ie are visible on the GUI, are framed in red, only back-end components are without a frame. }
	\label{fig:workflow}
\end{figure}


%%%%%
%
%1) Treating the image 
%- multiplying to enhance contrast
%- gamma correction
%- adaptive thresholding
%- inverting image
%
%2) User draws a window around root of interest, samples 5-10 points of chosen root.
%Using the mean value and standard deviation of the pixel intensity of these points a thresholding range is found. Say [mu-3sigma,mu+3sigma]. 
%(Using the same threshold for all the images in the experiment was not successful as the noise patterns of the images vary too much.)
%
%
%3) Treating the cropped image using different filters
%
%Approach 1: Colour separation filtering 
%- based on RGB values of points
%- gray scales image
%
%Approach 2: Brightness filtering (intensity-based approach)
%- enhances brightness 
%- eliminates too bright spots
%
%4) SKELETONIZATION
%- unified approach (approach 1+2)
%
%5) Optional: additional cleaning
%- iteratively removing bigger and bigger connected objects (in steps of 30 pixels)



%%%%%%%%%%%%%%%%%%%%%%%%%%%%%%%%%%%%%%%%%%%%%%%%%%%%%%%%%%%%%%%%%%%%%%%%%%%%%%%
%%%%%%%-----WORK FROM HERE----%%%%%%%%%%%




%%%%%%%%%%%%%%%%%-----INCLUDE THIS------------
%%----------------------------------------------------------------------------------------
%\subsection{Loading a file}
%
%%----------------------------------------------------------------------------------------
%\subsection{Zooming in}
%
%and cropping
%
%
%%----------------------------------------------------------------------------------------
%\subsection{Choosing points}
%
%
%%----------------------------------------------------------------------------------------
%\subsection{Processing the image}
%
%Two approaches:
%1.
%Converted into gray channel images.
%2.
%
%
%%----------------------------------------------------------------------------------------
%\subsection{Optional image cropping}
%
%Additional cropping of the root ultimately solves the problem of getting rid of any artificial branches caused by noise close to the tip of the root.
%Also, the cropping does not have to be very accurate.
%
%
%%----------------------------------------------------------------------------------------
%\subsection{Extracting the skeleton}
%
%%----------------------------------------------------------------------------------------
%\subsection{Optional skeleton cropping}
%
%
%%----------------------------------------------------------------------------------------
%\subsection{Optional foreign object removal}
%
%Other than that, the user can perform iterative cleaning that gradually gets rid of those artificial branches (most of the times) as well; one has to be careful not to "loose" the root completely, but the user can undo the last cleaning step. 
%
%%----------------------------------------------------------------------------------------
%\subsection{Optional tip forcing}
%
%
%%----------------------------------------------------------------------------------------
%\subsection{Final preparation}
%
%
%%----------------------------------------------------------------------------------------
%\subsection{Angle computation}
%%%%%%%%%%%%%%%%%%%%%%%------INCLUDE THIS


%0) Convert the image from rgb to gray scale (matlab: rgb2gray) (3-tensor to matrix)
%
%intensify the matrix \arrowvert distinguish roots for user
%highlight window \arrowvert help user where approximately roots are

%
%1. The user will select a region of interest on the image so only roots appear. This window can be used over the whole experiment (as camera doesn’t move) ///TODO: or better to define window for each root?
%
%2. The user then clicks on some points over the roots (I'd say at least 7, preferably 10). Using the mean value and standard deviation of the pixel intensity of these points a thresholding range is found. Say [mu-2sigma,mu+2sigma]. 
%Initially trying to use the same threshold for all the images in the experiment, maybe this will do the job already…
%
%3. After segmenting the images, the binary images need to postprocessed so we get rid of the bright spots and noise, eg by removing any object that has fewer than 1000 pixels (matlab: bwareaopen enough?). Then any holes that might appear in the roots need to be filled (matlab: imfill)

%
%4. Hopefully obtaining the skeleton version of the binary image (matlab: bwmorph).
%
%5. Label each individual object in the image so they can later be processed independently (matlab: bwconncomp).

%6. For each root we then get the list of point belonging to the tip (upper x- and y-values) and we compute the curvature (eg on a 3-point polygonal chain)


%----------------------------------------------------------------------------------------
%----------------------------------------------------------------------------------------
\section{Components of the GUI}

With the development of the GUI for more user-friendliness, we split our source code of the tool based on functionality, ie we create separate functions or modules that we then connect to several objects in the GUI. This process, also known as \textit{modularity} in software engineering, has various benefits for developing and maintaining the application: The code is less cluttered but more structured and readable and just by reading the main functions which calls all other subfunctions you get a general overview and understanding of what the code does. Also, it reduces redundancy in the main code if the codes get split up into smaller subfunctions and helps debugging. Speaking variables also add to a better understanding of the code.

%MODULARITY — main code is not as cluttered, things that are called again and again, migrate it. several functions use same code. 
%REDUCE REDUNDANCY
%MORE READIBLE, CLEARER, MORE STRUCTURED, DON’T GET LOST IF YOU ONLY READ MAIN CODE, GENERAL THIGNS THAT HAPPEN.
%SPEAKING VARIABLES. 

[INSERT FIGURE FRONT-END BACK-END HERE]

%Beyond that, provided the interface, ie input and output of each modules, is well defined modularity allows a team of developers to work on the functionality of one single module without involving other modules. They are not distracted by functionalities that are not relevant, known as information hiding in software engineering. Also, it helps to keep the code short and simple and focused one core functionality. It is also useful to identify bugs early on in the development process.

Here, it allowed us to not only handle front-end and back-end but also the different steps in the pipeline separately: 

Every function representing one specific step in the pipeline or objects accessed by various functions is encoded in the back-end which is not visible and relevant for the future user. These modules are connected via callbacks to different objects in the GUI which represent the front-end.

%Back-end as well as front-end needs maintaining. 
%The line between the two of them is often blurry.
%
%Front-end: involved with what the user sees, including design. 


Table [INSERT REFERENCE HERE] gives an overview of the different components and modules our software tool \textit{RootSkel} consists of. It has been developed over a cycle of various iterations, together with an exemplary future user. 
The current version of the package can be downloaded from [INSERT REFERENCE HERE]. %https://github.com/burfel/root-tip-angle/tree/master/src/Root_image_GUI_v1_5 and will be maintained on [INSERT GIOVANNIS GITHUB HERE].


\begin{figure}[h!]
	\centering
	\includegraphics[width=\textwidth]{../Figures/components.pdf}
	\caption{The different components and modules of \textit{RootSkel} with a description of each of them; the most important ones containing the core functionality of \textit{RootSkel} are framed in red, the high-level components are shaded in yellow.}
	\label{fig:modules}
\end{figure}


%image_process.m
%* extracts the cropped image 
%* extracts the colours from the sample pixels, averages it with a certain neighbourhood
%* takes a brightness range, an average of the three filters used 
%* approach 1: using input from the user to estimate the range of the root colours
%* initial colour filtering based on the RGB values of the selected points
%* subsequent gray filtering
%* RGB level difference based filtering
%* …….
%* approach 2: using brightness filter (intensity based)
%* Idea: We carefully enhance the brightness and then get rid of too bright spots.
%* gray scaling
%* creating a filtering mask according to the brightness rank



%MAYBE USE STH FROM HERE BUT COMMENTED OUT FOR NOW, AS WIDELY REPETITION:
%Even though most of image processing work is empirical, ie based on trial and error and educated guesses on the given data set, we try to explain why certain features were necessary to implement. %this explanations are marked in italics.
%The pipeline can be seen in the at the bottom of the GUI in figure [INSERT REFERENCE HERE].
%
%[INCLUDE IN GRAPHIC IF EXCEED WORD COUNT]
%
%%1. Open file
%%2. Zoom in
%%3. Choose Points
%%4. Process image
%%5. Crop image (optional): 
%%6. Get skeleton
%%7. Crop skeleton (optional): It allows the user to crop a region around the skeleton by hand to definitely remove things outside the region of interest. The user can redo the cropping in case she is not satisfied. 
%%8. Foreign object removal (optional): It allows the user to remove objects, ie if the skeleton contains undesired branches. The user can undo the last cleaning step in case the root gets removed in the last step, to not start the entire process again. 
%%9. Force tip (optional)
%%10. Final preparation
%%
%%The bar on the left shows various other buttons that allow the user to perform other functions:
%%* In the upper part: 
%%* Save variables
%%* Save figures: select the cropped image or the skeleton; both are greyed out as longs as objects have not been created
%%* Load variables
%%* Load figure: choose one of the radio buttons either on the left or on the right
%%* In the middle part: 
%%* Show skeleton; greyed out as long as the skeleton has not been created
%%* Clear figure: choose one of the radio buttons either on the left or on the right
%%* The bottom part to compute and display the angle:
%%* Using user’s point
%%* Using curvature
%
%The GUI includes helpful text along the pipeline as well as pop up windows including instructions or warnings in case the user does not follow them. 
%
%In the appendix [INSERT REFERENCE HERE] we guide the user through the different steps of the pipeline and present one example each; this can serve as a manual for users even though the steps are rather self-explanatory by just using the GUI. Also, having a GUI rather than a script with pop-up-windows not only makes the tool more visually attractive but it one can also repeat a step without restarting the whole process or script again. This all adds to the user-friendliness.

%----------------------------------------------------------------------------------------
%----------------------------------------------------------------------------------------
\section{Validation: Comparing the automated calculated angles with the manually calculated angles}

As a preview of validation we performed our webtool on 3 randomly chosen roots from 4 different experiments over a time of 330 minutes. We compared both the automatically computed angles with the previously computed manual angle. As a second validation step we also compared the angles to the computed angles with the turning point as user input.

Figure [INSERT HERE] shows the comparison of the angle of interest \( \Theta \), both of the manual as well as the automatically computed angle at about 11 times steps over a period of 330 minutes.

%\begin{figure}[h]
%\centering
%\includegraphics[width=0.6\textwidth]{Figures/2018-04-24.png}
%\caption{}
%\label{fig:2018-04-24}
%\end{figure}

%\begin{figure}
%	\begin{subfigure}
%		\includegraphics[]{../Figures/2018-04-24.png}
%	\end{subfigure}
%	\begin{subfigure}
%		\includegraphics[]{../Figures/2018-03-07.png}
%	\end{subfigure}
%	\begin{subfigure}
%		\includegraphics[width=\textwidth]{../Figures/2018-02-28.png}
%	\end{subfigure}
%\end{figure}

%\newpage

\begin{figure}[h!]

\centering

\includegraphics[width=.7\textwidth]{../Figures/2018-02-28.png}
%\includegraphics[width=150mm,scale=15]{../Figures/2018-02-28.png}

\includegraphics[width=.7\textwidth]{../Figures/2018-03-07.png}

\includegraphics[width=.7\textwidth]{../Figures/2018-04-24.png}

	\caption{Comparing the manually computed angle and the angle(s) computed by \textit{RootSkel}; in blue: computed by \textit{RootSkel}, in red: computed by \textit{RootSkel} with the turning point as user input, in yellow: manually computed angle. Each plot is labelled by the data set it was taken from (named after date) and the root number (starting from the left hand side in an image). Computations of all three approaches exhibit similar patterns; the angle \( \Theta \) continuously decreases. %Differences between the three approaches are in the magnitute of 5 degrees, single points vary up to 12 degrees, see also table [INSERT REFERENCE HERE].
		}

\end{figure}


Differences between the three approaches are in the magnitute of 5 degrees, single points vary up to 12 degrees, see also figure [INSERT REFERENCE HERE].
Table [INSERT REFERENCE HERE] compares the automatically computed angle by \textit{RootSkel} with the manually computed angles on the 3 roots. We compute the absolute difference in the angles and introduce an \textit{improvement score} defined as 
\[
 \text{score}_{\textbf{imp}} := | \frac{ \alpha_{\text{RootSkel}} - \alpha_{\text{manual}} }{\alpha_{\text{manual}}} |
\]
where $\alpha_{\text{RootSkel}}$ denotes the angle computed by \textit{RootSkel} and $\alpha_{\text{manual}}$ denotes the manually computed angle; this way we compare to the manually computed angles. For how much our method is really an improvement of the theoretical true value, that cannot be computed as any measurement will be subject to errors, we refer to the discussion section [INSERT REFERENCE HERE]. The improvement score on our 3 example data sets average between 5\% and 23\% with an ensemble average of 12.4\%, which can be concluded from figure [INSERT REFERENCE HERE].

When studying our data sets (manually computed and automatically computed angles), we could oberve that as the angle approaches zero, the values get more volatile, ie the variance increases; this matches up with our observations in presented in figure [INSERT REFERENCE HERE]. However, it should be noted that our improvement score is based on relative values and, therefore, can easily blow up for small values. %(of manually computed angles and increasing or constant difference/ variance to the automatically computed angle).


\begin{figure}[h!]
	
	\centering
	
	\includegraphics[width=.7\textwidth]{../Figures/2018-02-28-stats-new.png}
	%\includegraphics[width=150mm,scale=15]{../Figures/2018-02-28.png}
	
	\includegraphics[width=.7\textwidth]{../Figures/2018-03-07-stats-new.png}
	
	\includegraphics[width=.7\textwidth]{../Figures/2018-04-24-stats-new.png}
	
	\caption{Comparing the autmoatically computed angle by \textit{RootSkel} with the manually computed angles on the 3 roots. Average values of the difference in the angles and the improvement score are highlighted by a read margin; values that highly skew the results and are advised to be left out when doing the average calculations are highlighted in yellow.
		The p-values of a two sample t-test are highlighted in green.\\ 
		It should be noted that we made an educated guess of the measurement error, i.e. the total error, to be about about 5\% of our computed \textit{RootSkel} value. We round this error to two significant digits according to convention and computed the absolute difference in the angles and the improvement score with this precision; we only round once we present the values to avoid rounding errors. }
	
\end{figure}


As a validation that the computed values of \textit{RootSkel} are similar enough to the manually computed values, we perform a t-test between the two sets of angles.
In all three cases we fail to reject the null hypothesis (at a 95\% confidence interval), implying that the difference between each of the two samples is not statitstically significant. The p-values of the two sample t-test can be found in figure [INSERT REFERENCE HERE].

%we fail to reject the null hypothesis. So we are not "rejecting the null", implying that we don't have enough evidence to assume that the difference in means is different from zero.

%We assume there is no difference, and since the t-score with p-value is larger than 0.05 we cannot reject this hypothesis, ie the difference between the two samples is not statistically significant; this does not mean there is no difference that can be relevant in statistics.


%We show that there is no significant difference between the two sets of values. 

%Kolmogorov Smirnov test is independent of distribution and would be suitable for your data.

%FORVIVA: Alternatively, we could ask how similar the two data sets are and perform a \( \Chi^{2} \) test; 
% tests for strenght of association between the two variables.
%If it was due to chance we would see it in 23\%.


%The variance of the angle pairs is shown in figure [INSERT HERE]. Performing a t-test we can conclude that the improvement of the angle computation accuracy is significant. Approximately [INSERT HERE] \% due to human error can be eliminated.
%%Looking at the variance of the angle pairs and performing a t-Test, we can conclude that by our tool we can minimise the human error by [INSERT HERE]\%. 
%
%Once more angles have been compared, more confident statistics can be obtained. 
%
%[INSERT FIGURE HERE]




% Chapter 4: DISCUSSION

\chapter{Discussion} % Main chapter title

\label{discussion} % For referencing the chapter elsewhere, use \ref{Chapter1} 

%%----------------------------------------------------------------------------------------
%
%% Define some commands to keep the formatting separated from the content 
%\newcommand{\keyword}[1]{\textbf{#1}}
%\newcommand{\tabhead}[1]{\textbf{#1}}
%\newcommand{\code}[1]{\texttt{#1}}
%\newcommand{\file}[1]{\texttt{\bfseries#1}}
%\newcommand{\option}[1]{\texttt{\itshape#1}}

%----------------------------------------------------------------------------------------

PUT RESULTS IN BROADER PICTURE

%* What informed choice of methods?
%* Could it have been done in another way?
%* Which aspects of the work could be taken further?

%----------------------------------------------------------------------------------------
%----------------------------------------------------------------------------------------
\section{Challenges and limitations}

%Difficult decisions had to be made in planning the research, leading to subsequent tradeoffs. Then, focus in on the need for future work.
%----------------------------------------------------------------------------------------
\subsection{Low contrast}

There is a very low contrast between the background and the roots, so that one could hardly recognise the roots on some images [INSERT EXAMPLE IMAGE HERE].
Inverting the image was not enough. We intensified the contrast by ... / making the background darker, we increased the brightness. We used further filtering so that the difference between two pixels is more pronounced.
It was very much a trial \& error process.

Classify only things without structure as noise. 
Noise was subtle/ hiding and could only be seen on filtered images.

Many objects of no interest which can be not 


[INSERT 4 EXAMPLES HOW DIFFICULT DATA WAS]


%----------------------------------------------------------------------------------------
\subsection{Objects interferring with objet of interest}

Eg specs not so much of an issue as separate from the root.
Issue when it is connected.


%----------------------------------------------------------------------------------------
\subsection{Skeletonisation}
Get rid of loop.

How to go along the curve?


%----------------------------------------------------------------------------------------
\subsection{Scalability}

A lot of paramter tuning to single images. Work with single images to tune.


%----------------------------------------------------------------------------------------
\subsection{Reproducability}

angle computation yes, but still variability in pre-processing step


%----------------------------------------------------------------------------------------
\subsection{Highly pixeled images}
Due to compressing? 
Challenge in curvature computing part
Eg using Hough transform from Matlab Image Processing Toolbox to detect lines in an image not applicable here as highly pixled image.

% GIVE SAMPLE IMAGE

%----------------------------------------------------------------------------------------
\subsection{Problems}
A lot of trial \& error / hand-picking. 


NOISE PATTERN VARIES A LOT ACROSS IMAGES.

%----------------------------------------------------------------------------------------
\subsection{High user-interaction}
Might be reduced with better-quality images, however very flexible.


standard-deviation not as high as i think?

Would be desirable if it could be automated more. 

%----------------------------------------------------------------------------------------
%----------------------------------------------------------------------------------------
\section{Suggestions for future data acquisition}


%----------------------------------------------------------------------------------------
\subsection{More focus on images}


%----------------------------------------------------------------------------------------
\subsection{Resolution}

Better camera, less waste of resolution

improve spatial resolution, other format? no compression?
 
 
%----------------------------------------------------------------------------------------
%----------------------------------------------------------------------------------------
\section{Broader application of this tool}

This tool can be reused for many purposes, it is not restricted to root detection.
Might also be used for easier problems like gravitropism.


%----------------------------------------------------------------------------------------
%----------------------------------------------------------------------------------------
\section{Further work}

Here in the first version of the tool, the main goal was to standardise the angle computation; if in the future  a method for handling the different noise pattern in the images was efficiently handled which require less user input, it would be desireable to automate the whole angle computation.

Use on better-quality images in the future.

%High-throughput software tools that can produce objective, quantitative analyses od the resulting images are now required.

graph on GUI will be implemented, to make sure that we are computing the right angle.


- How tool accessible for people who do not have matlab?

NowL GUI reenter, no single script. no time, just did not bother, can easily be added.



\subsection{Other programming languages}

Alternative languages that were considered were \textit{Python} and \textit{Julia}. Another recommended language is \textit{OpenCV} as it is very fast and well documented. Other non-open source software such as \textit{ImageJ}, a Java based image processing program, and \textit{Avizo} which is a general-purpose commercial software application for scientific and industrial data visualisation and analysis with a nice GUI, could not be investigated further in this work. 


%----------------------------------------------------------------------------------------
\subsection{Other ways of curvature and angle measuring}

What was implemented as we found that this approach worked best on these data and this resolution.

%Standardised and automated version of angle computation
\subsection{Othere definition of the angle}
Approaches like computing the Gaussian mean curvature, definitions of different, possibly more robust methods of computing and angle %are mentioned here, but has 
have not been incorporated in our final tool. 


\subsection{Root is not in plane}

compared to gravitropism, it is much harder to keep the roots in a plane. we only capture the angle based on 2D images.

\subsection{Curvature better on dense data set}

% FUTURE WORK: curvature on a sparse polygon like this might not be very
% meaningful but that does not mean it is not possible to apply the
% discrete approximation to the derivative to compute it. 
% The same code on a more densly samples outline will give a good
% approximation of the curvature of the outline


\subsection{Less user interaction / automatisation}

%----------------------------------------------------------------------------------------
\subsubsection{Adaptive thresholding}
%  Another similar approach would be to ask the user to quickly draw small ROI (rectangles) to frame each root in the inverted (much easier) image. The code could then global threshold each ROI separately and go from there.
Before opting to take user input in the form of samples of the roots in the image, we investigated adaptive (global) thresholding on each of the root and an adaptive variable setting approach on all of the roots together to extract the root. This however failed, or would have been beyond the scope of this project -- the reason why we implemented it the way it was suggested.



\subsection{more data}


The workflow developed via  many iterations on different images and 

elaborate pre-procecessing tool for skeletonisation

high functionality, reiterated process

has been overengineered on one dataset, if it should be robust on other data, it needs to be trained/ tweeked on other data.




\section{Next steps}

\subsection{Maintenance and building upon}

\subsection{Features to implement}

 
% Chapter 5: CONCLUSION

\chapter{Conclusion} % Main chapter title

\label{conclusion} % For referencing the chapter elsewhere, use \ref{Chapter1} 

%%----------------------------------------------------------------------------------------
%
%% Define some commands to keep the formatting separated from the content 
%\newcommand{\keyword}[1]{\textbf{#1}}
%\newcommand{\tabhead}[1]{\textbf{#1}}
%\newcommand{\code}[1]{\texttt{#1}}
%\newcommand{\file}[1]{\texttt{\bfseries#1}}
%\newcommand{\option}[1]{\texttt{itshape#1}}

%%----------------------------------------------------------------------------------------

%WHAT, WHY, WHY ADVANTAGE, WHAT WAS CHALLENGE?


%%----------------------------------------------------------------------------------------


%Here we presented \textit{RootSkel} -- a novel and stand-alone image-analysis-based software tool developed in MATLAB optimised for noise-intensive electrotropism images that can compute the curvature and angle at the root tip in a standardised fashion with a user-friendly and very flexible pre-processing step to extract the skeleton of the root from possibly very noise image data sets.

Here we presented \textit{RootSkel} -- a novel stand-alone and intuitive image-analysis-based software tool developed in MATLAB to compute the curvature and angle at plant root tips in a standardised fashion. It was designed to work with noise-intensive electrotropism images and comes with a user-friendly and flexible pre-processing step to extract the skeleton of the root. % from possibly very noise image data sets.

The software has been designed using an extensive amount of different filtering techniques optimised on the image data set described in this work and can therefore be used to work with standard images from consumer digital cameras. 

Automated image capturing as well as the design of the software presented here both aim to reduce the time-consuming process of the biologist quantifying the root tip curvature manually but more importantly, it standardises the computations and makes the results reproducible and comparable over a large amount of data. It offers the possibility to extend the analysis by including different angle definitions to capture the curvature of the plant root and compare them. Also, the software tool is not limited to computing the angle of root tips but due to the flexibility of the pre-processing step can be used for any other curved or polynomial-like structure. 

Angles computed by \textit{RootSkel} seem to correlate well with equivalent manual measures; based on angles of 3 Arabidopsis roots we showed that the computed by \textit{RootSkel} do not differ significantly (\(P >0.05\)) from the manually compted ones. % and, therefore, represent a good emulation thereof. 

However, the robustness and reproducibility of \textit{RootSkel} need to be further validated; suggestions for future work can be implemented. 

We hope \textit{RootSkel} will contribute to the understanding of highly complex and poorly-understood phenomenons like eletrotropism in plants and possibly other tropisms. %, as well as plant growth in general.

%.FROM ABSTRACT
%Unlike when doing the angle computation to measure the curvature of a root tip manually, our tool ensures a standardised version of the angle computation. On top of it, we  developed a user-friendly graphical user interface (GUI) that will help the user in processing the images and compute the angles in a standardised and controlled fashion.
%
%We use a data set of high-throughput time-lapse images of Arabidopsis roots from 5 experiments with 4-5 roots each containing between 32 and 36 images over a period of approximately 5 hours produced by the Plant Morphology lab at Imperial College London, we evaluate our results by comparing it to previously manually computed angles. [INCLUDE RESULTS A LA 70\% of the images could be found to be reproduced]


%HHope to be able to standardise it in the future.

 

%----------------------------------------------------------------------------------------
%	BIBLIOGRAPHY
%----------------------------------------------------------------------------------------

\printbibliography[heading=bibintoc]

%----------------------------------------------------------------------------------------

%----------------------------------------------------------------------------------------
%	THESIS CONTENT - APPENDICES
%----------------------------------------------------------------------------------------

\appendix % Cue to tell LaTeX that the following "chapters" are Appendices

% Include the appendices of the thesis as separate files from the Appendices folder
% Uncomment the lines as you write the Appendices

%% Appendix A


\chapter{On the data set}

The data set used can be found on [INSERT REFERENCE GITHUB HERE].

%----------------------------------------------------------------------------------------
%----------------------------------------------------------------------------------------
\section{Suggestions for future data acquisition}

We will give some suggestions on how to improve the quality of the images in the future.


%----------------------------------------------------------------------------------------
\subsection{Stable conditions in the experiment}

In general, one should ensure to keep the conditions over one experiment stable as far as possible. It should be noted that this is far harder to achieve in a dynamic system as the one to capture electrotropism compared to a gravitropism setup usually done in an agar gel. 
This includes a constant number of tubes and roots, constant water level or surface and no external movement to the roots or the tubes to keep the noise distribution as constant as possible. Roots that are affected by change in reflections or illumination (eg by people walking into the room) have been proven to be very hard to handle and are often lost in the preprocessing step. 
Keeping the medium clean of dirt and bubbles as far as possible also will improve the preprocessing step. 


\subsection{No objects interfering with the object of interest}

The experimentalist should make sure that there are no objects interfering with the object of interest, be it other roots or any other similar looking objects that are hard to discern even by eye, throughout the experiment.

%llumination different — challenge to get background right
%local thresholds maybe?


%----------------------------------------------------------------------------------------
\subsection{Increase of the contrast}

It is advisable to work towards achieving a higher contrast in the images, so roots can be clearly distinguished from the background, also by eye.


%%----------------------------------------------------------------------------------------
%\subsection{More focus on images}


%----------------------------------------------------------------------------------------
\subsection{Higher resolution}


As the highly pixeled nature of the images did cause problems also in the angle computation, it is recommendable to try to increase the resolution of the images. We could also try to zoom into the roots or even one single root so our actual object(s) of interest take a bigger fraction in the images instead of wasting resolution on things that are of no interest. 

It could also be the compression step after taking the images that might cause or contribute to the low resolution of the images. One could attempt to user other formats to save the images. 

As a last suggestion, different cameras could be compared to see if it does have an effect on the quality of the images. 

%Better camera, less waste of resolution
%improve spatial resolution, other format? no compression?



%----------------------------------------------------------------------------------------
%----------------------------------------------------------------------------------------
%----------------------------------------------------------------------------------------
\chapter{Manual of \textit{RootSkel}}


%%%%%%%%%%%%%%%%%%%%%%%%%%%%%%%%%%%%%%%%%%%%%
%\chapter{Data}
%
%The data was taken by a ... camera using a raspberry Pi.
%Insert details regarding data collection.
%
%\chapter{Frequently Asked Questions} % Main appendix title
%
%\label{AppendixA} % For referencing this appendix elsewhere, use \ref{AppendixA}
%
%\section{How do I change the colors of links?}
%
%The color of links can be changed to your liking using:
%
%{\small\verb!\hypersetup{urlcolor=red}!}, or
%
%{\small\verb!\hypersetup{citecolor=green}!}, or
%
%{\small\verb!\hypersetup{allcolor=blue}!}.
%
%\noindent If you want to completely hide the links, you can use:
%
%{\small\verb!\hypersetup{allcolors=.}!}, or even better: 
%
%{\small\verb!\hypersetup{hidelinks}!}.
%
%\noindent If you want to have obvious links in the PDF but not the printed text, use:
%
%{\small\verb!\hypersetup{colorlinks=false}!}.

%\include{Appendices/AppendixB}
%\include{Appendices/AppendixC}

\end{document}  
