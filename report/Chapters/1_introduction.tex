% Chapter 1: INTRODUCTION

\chapter{Introduction} % Main chapter title

\label{introduction} % For referencing the chapter elsewhere, use \ref{Chapter1} 

%----------------------------------------------------------------------------------------

% Define some commands to keep the formatting separated from the content 
\newcommand{\keyword}[1]{\textbf{#1}}
\newcommand{\tabhead}[1]{\textbf{#1}}
\newcommand{\code}[1]{\texttt{#1}}
\newcommand{\file}[1]{\texttt{\bfseries#1}}
\newcommand{\option}[1]{\texttt{\itshape#1}}

%----------------------------------------------------------------------------------------

RELEVANCE AND TOPICALITY OF THE AIMS AND OBJECTIVES AND MY CONTRIBUTION TO THE RESEARCH.
HOW HAS THE PROJECT ADVANCED THE FIELD?

%----------------------------------------------------------------------------------------

\section{Biological background}

An important biological phenomenon studied by plant morphologists is \textit{tropism} [ADD TO GLOSSARY], which is used to indicate the turning movement of a biological organism, here of a plant, when exposed to different environmental simuli [INSERT REFERENCE]. Usually the stimulus involved is added to the name, eg \textit{phototropism} as a reaction to sunlight; it can be either \textit{positive}, ie towards the stimulus, or \textit{negative}, ie away from the stimulus.
The most frequently observed and best studied tropism in \textit{gravitropism}, which describes the process of how plants grow as a response to gravity. It was firstly scientifically documented by Charles Darwin [INSERT REFERENCE] and can be observed in higher and many lower plants as well as other organisms [INSERT REFERENCE]: Roots show \textit{positive gravitropism}, ie they grow in the direction of the gravitational pull whereas stems grow in the opposite direction. An easy experiment to do is to lay a potted plant onto its side; over time the stem will begin to turn upwards and thus show negative gravitropism. \\
A far less studied process is \textit{electrotropism} which describes the growth or movement of a plant when exposed to an electric field and which was the tropism under study in this project. 

A high-overview explanation of the experiment setup and data collection can be found in section [INSERT REFERENCE TO SECTION METHOD].

%----------------------------------------------------------------------------------------

\section{Literature review}

\subsection{RootTrace}

RootTrace [INSERT REFERENCE] has been developed to measure root lengths across time serie image data; biologists have also used it to measuer highly curved roots. A graphical user interface (GUI) implemented within the RootTrace framework makes it easy for the user to handle. However, this tool has failed on our image data set [INSERT REFERENCE HERE], probably due to the high noise level found in electrotropism images compared to gravitropism images.






\subsection{PlantCV}


COULD HAVE USED ONE OF THESE APPRAOCHES, BUT STARTED FROM SCRATCH. BASED ON THIS DATA SET.


However, 


Electopism much much harder since photos noisier.



%%----------------------------------------------------------------------------------------
%%----------------------------------------------------------------------------------------
%%----------------------------------------------------------------------------------------
%
%\section{What this Template Includes}
%
%\subsection{Folders}
%
%\subsection{Files}
%
%\keyword{main.out} -- this is an auxiliary file generated by \LaTeX{}, if it is deleted \LaTeX{} simply regenerates it when you run the main \file{.tex} file.
%
%
%%----------------------------------------------------------------------------------------
%
%\section{Filling in Your Information in the \file{main.tex} File}\label{FillingFile}
%
%
%
%%----------------------------------------------------------------------------------------
%
%\section{The \code{main.tex} File Explained}
%
%
%
%%----------------------------------------------------------------------------------------
%
%\section{Thesis Features and Conventions}\label{ThesisConventions}
%
%
%\subsection{Printing Format}
%
%This thesis template is designed for double sided printing (i.e. content on the front and back of pages) as most theses are printed and bound this way. Switching to one sided printing is as simple as uncommenting the \option{oneside} option of the \code{documentclass} command at the top of the \file{main.tex} file. You may then wish to adjust the margins to suit specifications from your institution.
%
%The headers for the pages contain the page number on the outer side (so it is easy to flick through to the page you want) and the chapter name on the inner side.
%
%The text is set to 11 point by default with single line spacing, again, you can tune the text size and spacing should you want or need to using the options at the very start of \file{main.tex}. The spacing can be changed similarly by replacing the \option{singlespacing} with \option{onehalfspacing} or \option{doublespacing}.
%
%
%\subsection{References}
%
%The \code{biblatex} package is used to format the bibliography and inserts references such as this one \parencite{Reference1}. The options used in the \file{main.tex} file mean that the in-text citations of references are formatted with the author(s) listed with the date of the publication. Multiple references are separated by semicolons (e.g. \parencite{Reference2, Reference1}) and references with more than three authors only show the first author with \emph{et al.} indicating there are more authors (e.g. \parencite{Reference3}). This is done automatically for you. To see how you use references, have a look at the \file{Chapter1.tex} source file. Many reference managers allow you to simply drag the reference into the document as you type.
%
%Scientific references should come \emph{before} the punctuation mark if there is one (such as a comma or period). The same goes for footnotes\footnote{Such as this footnote, here down at the bottom of the page.}. You can change this but the most important thing is to keep the convention consistent throughout the thesis. Footnotes themselves should be full, descriptive sentences (beginning with a capital letter and ending with a full stop). The APA6 states: \enquote{Footnote numbers should be superscripted, [...], following any punctuation mark except a dash.} The Chicago manual of style states: \enquote{A note number should be placed at the end of a sentence or clause. The number follows any punctuation mark except the dash, which it precedes. It follows a closing parenthesis.}
%
%The bibliography is typeset with references listed in alphabetical order by the first author's last name. This is similar to the APA referencing style. To see how \LaTeX{} typesets the bibliography, have a look at the very end of this document (or just click on the reference number links in in-text citations).
%
%\subsubsection{A Note on bibtex}
%
%The bibtex backend used in the template by default does not correctly handle unicode character encoding (i.e. "international" characters). You may see a warning about this in the compilation log and, if your references contain unicode characters, they may not show up correctly or at all. The solution to this is to use the biber backend instead of the outdated bibtex backend. This is done by finding this in \file{main.tex}: \option{backend=bibtex} and changing it to \option{backend=biber}. You will then need to delete all auxiliary BibTeX files and navigate to the template directory in your terminal (command prompt). Once there, simply type \code{biber main} and biber will compile your bibliography. You can then compile \file{main.tex} as normal and your bibliography will be updated. An alternative is to set up your LaTeX editor to compile with biber instead of bibtex, see \href{http://tex.stackexchange.com/questions/154751/biblatex-with-biber-configuring-my-editor-to-avoid-undefined-citations/}{here} for how to do this for various editors.
%
%
%\begin{table}
%\caption{The effects of treatments X and Y on the four groups studied.}
%\label{tab:treatments}
%\centering
%\begin{tabular}{l l l}
%\toprule
%\tabhead{Groups} & \tabhead{Treatment X} & \tabhead{Treatment Y} \\
%\midrule
%1 & 0.2 & 0.8\\
%2 & 0.17 & 0.7\\
%3 & 0.24 & 0.75\\
%4 & 0.68 & 0.3\\
%\bottomrule\\
%\end{tabular}
%\end{table}
%
%You can reference tables with \verb|\ref{<label>}| where the label is defined within the table environment. See \file{Chapter1.tex} for an example of the label and citation (e.g. Table~\ref{tab:treatments}).
%
%\subsection{Figures}
%
%There will hopefully be many figures in your thesis (that should be placed in the \emph{Figures} folder). The way to insert figures into your thesis is to use a code template like this:
%\begin{verbatim}
%\begin{figure}
%\centering
%\includegraphics{Figures/Electron}
%\decoRule
%\caption[An Electron]{An electron (artist's impression).}
%\label{fig:Electron}
%\end{figure}
%\end{verbatim}
%Also look in the source file. Putting this code into the source file produces the picture of the electron that you can see in the figure below.
%
%\begin{figure}[th]
%\centering
%\includegraphics{Figures/Electron}
%\decoRule
%\caption[An Electron]{An electron (artist's impression).}
%\label{fig:Electron}
%\end{figure}
%
%Sometimes figures don't always appear where you write them in the source. The placement depends on how much space there is on the page for the figure. Sometimes there is not enough room to fit a figure directly where it should go (in relation to the text) and so \LaTeX{} puts it at the top of the next page. Positioning figures is the job of \LaTeX{} and so you should only worry about making them look good!
%
%Figures usually should have captions just in case you need to refer to them (such as in Figure~\ref{fig:Electron}). The \verb|\caption| command contains two parts, the first part, inside the square brackets is the title that will appear in the \emph{List of Figures}, and so should be short. The second part in the curly brackets should contain the longer and more descriptive caption text.
%
%The \verb|\decoRule| command is optional and simply puts an aesthetic horizontal line below the image. If you do this for one image, do it for all of them.
%
%\LaTeX{} is capable of using images in pdf, jpg and png format.
%
%
%%----------------------------------------------------------------------------------------
%
%\begin{flushright}
%Guide written by ---\\
%Sunil Patel: \href{http://www.sunilpatel.co.uk}{www.sunilpatel.co.uk}\\
%Vel: \href{http://www.LaTeXTemplates.com}{LaTeXTemplates.com}
%\end{flushright}
