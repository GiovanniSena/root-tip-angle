% Chapter 1: INTRODUCTION

\chapter{Introduction} % Main chapter title

\label{introduction} % For referencing the chapter elsewhere, use \ref{Chapter1} 

%----------------------------------------------------------------------------------------

% Define some commands to keep the formatting separated from the content 
\newcommand{\keyword}[1]{\textbf{#1}}
\newcommand{\tabhead}[1]{\textbf{#1}}
\newcommand{\code}[1]{\texttt{#1}}
\newcommand{\file}[1]{\texttt{\bfseries#1}}
\newcommand{\option}[1]{\texttt{\itshape#1}}

%----------------------------------------------------------------------------------------


In the following chapter we will briefly familiarise the reader with the biological background of this work, summarise recent literature regarding the problem at hand to put our work into context and show in which way it contributed to the field.
%RELEVANCE AND TOPICALITY OF THE AIMS AND OBJECTIVES AND MY CONTRIBUTION TO THE RESEARCH.
%HOW HAS THE PROJECT ADVANCED THE FIELD?

%----------------------------------------------------------------------------------------
%----------------------------------------------------------------------------------------
\section{Biological background}

An important biological phenomenon studied by plant morphologists is \textit{tropism} [ADD TO GLOSSARY], which is used to indicate the turning movement of a biological organism, here of a plant, when exposed to different environmental simuli [INSERT REFERENCE]. Usually the stimulus involved is added to the name, eg \textit{phototropism} as a reaction to sunlight; it can be either \textit{positive}, ie towards the stimulus, or \textit{negative}, ie away from the stimulus. 
Various types of tropism can be found in figure [INSERT REFERENCE HERE].

\begin{figure}[h]
	\centering
	\includegraphics[width=0.6\textwidth]{../Figures/tropism.png}
	\caption{Different forms of tropism: Phototropism, thigmotropism, hydrotropism, gravitropism and different effects of gravitropism; taken from [INSERT REFERENCE].}
	\label{fig:tropism}
\end{figure}

The most frequently observed and best studied tropism in \textit{gravitropism}, which describes the process of how plants grow as a response to gravity. It was firstly scientifically documented by Charles Darwin [INSERT REFERENCE] and can be observed in higher and many lower plants as well as other organisms [INSERT REFERENCE, INSERT FIGURE DIFFERENT TROPISMS FROM G SLIDES]: Roots show \textit{positive gravitropism}, ie they grow in the direction of the gravitational pull whereas stems grow in the opposite direction, see figure [INSERT REFERENCE HERE]. An easy experiment to do is to lay a potted plant onto its side; over time the stem will begin to turn upwards and thus show negative gravitropism. \\
A far less studied process is \textit{electrotropism} which describes the growth or movement of a plant when exposed to an electric field and which was the tropism under study in this project. 

A high-overview explanation of the experiment setup and data collection to study electrotropism can be found in section [INSERT REFERENCE TO SECTION METHOD].

%----------------------------------------------------------------------------------------
%----------------------------------------------------------------------------------------
\section{Literature review}

The majority of traditional root development bioassays only consider a small number of points [REFEREENCE PERRY ET AL, 2001 AND MAIN ROOTTRACE PAPER]. They are informative in terms of long-term effects on root growth; however, small and temporary changes can not be captured [REFERENCE MAIN ROOTTRACE].
Recently developed tools have considered a higher number of time points which allows a better study of how the growth process develps and the plant's responses [REFERENCE MAIN ROOTTRACE; ISHIKAWA AND EVANS, 1997; VAN DER WEELE ER AL, 2003; CHAVARRIA-KRAUSER ET AL, 2007; MILLER ET AL, 2007].  

Image sequences can contain a representative "snapshot" description of a plant's developmental stage; also they can be generated at high speed [REFERENCE TO EXPERIMENT SETUP IN CHAPTER ....]. From manual time-lapse photography [REFERENCE VAN DER LAAN, 1934; MICHENER, 1938] to measure lengths of seedlings after applying different external stimuli, digital camera  technology has improved and low digital storage cost has become available which has made it comparatively easy to collect large, time-stamped digital image data sets monitoring root growth [REFERENCE MAIN ROOTTRACE].  
Once the root growth changes have been extracted from the imge data, one can correlate their timing with the impact of different external signals including hormonal and environmental on processes such as cell division and cell expansion [REFERENCE MAIN ROOTTRACE]. This will help to understand root development in general. 

Analysing image data manually, however, is time-consuming, very subjective and thus error-prone [REFERENCE MAIN ROOTTRACE]. When the analysis is done "by eye", it becomes difficult to reproduce measurements as they are not standardised by any automatised approach, ie made objective, and subject to human bias. Also, subtle phenotypes, ssuch as a delay in the response, might be missed [REFERENCE ROOTTRACE]. 
%Therefore, high-throughput software tools that can produce objective, quantitative analyses od the resulting images are now required.

Different groups have developed tools for gravitropism and have shown the power of automatised image-analysis techniques compared to manual methods. An overview was created in figure [INSERT REFERENCE HERE].


%----------------------------------------------------------------------------------------
%----------------------------------------------------------------------------------------
\section{Motivation}
The work described here is motivated by mainly three factors: definition and standardisation  or objectivity of the so far manually computed angle, flexibility and user-friendliness in the pre-processing step, and adaptability to standard consumer cameras. The software tool was designed to be used by a user, typically a biologist, with no need of specific knowledge in image processing nor plant morphology.

%----------------------------------------------------------------------------------------
\subsection{RootTrace}

RootTrace [INSERT REFERENCE] has been developed to measure root lengths across time serie image data; biologists have also used it to measuer highly curved roots. A graphical user interface (GUI) implemented within the RootTrace framework makes it easy for the user to handle. However, this tool has failed on our image data set [INSERT REFERENCE HERE], probably due to the high noise level found in electrotropism images compared to gravitropism images.


%\subsection{PlantCV}
%
%COULD HAVE USED ONE OF THESE APPRAOCHES, BUT STARTED FROM SCRATCH. BASED ON THIS DATA SET.
%
%
%However, 
%Electopism much much harder since photos noisier.

