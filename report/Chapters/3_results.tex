% Chapter 3: RESULTS

\chapter{Results} % Main chapter title

\label{results} % For referencing the chapter elsewhere, use \ref{Chapter1} 

%%----------------------------------------------------------------------------------------
%
%% Define some commands to keep the formatting separated from the content 
%\newcommand{\keyword}[1]{\textbf{#1}}
%\newcommand{\tabhead}[1]{\textbf{#1}}
%\newcommand{\code}[1]{\texttt{#1}}
%\newcommand{\file}[1]{\texttt{\bfseries#1}}
%\newcommand{\option}[1]{\texttt{\itshape#1}}

%----------------------------------------------------------------------------------------

PROVIDE A HIGH-LEVEL REVIEW, NOT LOTS OF NUMBERS

REPRODUCABILITY OF WORK, ie make sure reader could redo it

%%----------------------------------------------------------------------------------------
%%----------------------------------------------------------------------------------------

The following section of this report will explain the tool briefly from a technical perspective and show how to use the tool in practice. Additionally, we present an example of the tool in use.

The code is open-source and publically available on [INSERT GITHUB REFERENCE HERE]; all previous versions including log files can be found on [INSERT GITHUB REFERENCE HERE].
[EMPHAISE THIS: MAKE RESEARACH TRANSPARENT]

\section{Key components}

Figure [INSERT REFERENCE HERE] explains the key components of this image-analysis software tool to address the problem of highly noisy electrotropism consumer camera images of Arabidopsis roots and a standardised way of computing the angle for the curved root tip. 

This tool takes the form of a MATLAB program with a graphical user interface.

\section{Elaborate pre-processing tool for skeletonisation}
high functionality, reiterated process

%----------------------------------------------------------------------------------------

\section{Optimisation: Graphical User Interface (GUI)}

\begin{itemize}
	\item User interaction
	\item Error messages if user does not enter right values.
\end{itemize}

INCLUDE FIGURES HERE.

%----------------------------------------------------------------------------------------

\section{Different ways of curvature and angle measuring}

What was implemented as we found that this approach worked best on these data and this resolution.

%----------------------------------------------------------------------------------------

\section{Standardised and automated version of angle computation}

%----------------------------------------------------------------------------------------

\section{Comparing the automated calculated angles with the manually calculated angles}

%----------------------------------------------------------------------------------------

Two approaches:
1.
Converted into gray channel images.
2.

%----------------------------------------------------------------------------------------

\section{Learning \LaTeX{}}


\subsection{A (not so short) Introduction to \LaTeX{}}


\subsection{A Short Math Guide for \LaTeX{}}


\subsection{Common \LaTeX{} Math Symbols}


\subsection{\LaTeX{} on a Mac}
 
%----------------------------------------------------------------------------------------

\section{Getting Started with this Template}


\subsection{About this Template}


